\section{An Overview of Rust}
\label{sec:rust}

Rust is a multi-paradigm, general-purpose language. It draws from several previous languages including C++, Haskell, and Standard ML. While often referred to as a systems programming language, its usage is spreading rapidly to other areas including to some scientific programming disciplines~\cite{nature.rust.2020}.

The Rust language began in 2006 as a personal project of Graydon Hoare, an employee of the Mozilla Corporation, with Mozilla beginning to sponsor the work in 2009 and officially announcing the project in 2010~\cite{asay.2021}. The first pre-alpha numbered version of the compiler was Rust 0.1, which was released in January of 2012. The current (as of this writing) version of Rust is 1.63.0 and was released in August of 2022.

This section will briefly highlight some basic elements of the Rust language so as to help in understanding later code samples.

\subsection{Variables and Data Types}

Rust variables are declared with the \texttt{let} keyword. By default, a Rust variable is declared as an immutable value. This can be changed by including the \texttt{mut} keyword in the declaration:

\begin{lstlisting}
// This value is immutable:
let value = 0;
\end{lstlisting}
\begin{lstlisting}
// This value can be changed:
let mut value = 0;
\end{lstlisting}

A variable's declaration can include a type specification, but it is not required in cases where the type can be inferred from the right-hand side of the expression.

The basic types provided by Rust include a range of fixed-width numeric types covering signed and unsigned integers, float-point numbers, Boolean values, and characters. Rust also supports pointers to data, using the \texttt{\&} character to indicate a pointer value rather than a direct value. Pointers may be immutable or mutable, and the data to which they point may be declared as mutable or immutable.

There are two compound types supported: the \textit{tuple} and the \textit{array}. These differ mainly in that a tuple's elements are not required to be all of the same type, whereas an array's elements must all be of the same type.

Lastly, Rust also provides support for structured types, enumerated types, and functions as first-class data types. Additionally, the standard library includes a range of common collection types, such as vectors, strings, and hash-maps.

\subsection{General Syntax}

Syntactically, Rust most resembles C and C++. However, much of Rust's design is more influenced by functional programming principles~\cite[Chapter~13]{rpl.2019}. It organizes code into blocks using curly brackets and features typical control flow keywords such as \texttt{if}, \texttt{else}, \texttt{while}, and \texttt{for}. But it also treats almost every part of a function body as an expression, including the control flow operators~\cite[Chapter~6]{programming.rust.2021}. For example, the ternary operator of C is not provided but because an \texttt{if}-block is an expression one can get the same effect in code:

\begin{lstlisting}[language=C]
// Ternary in C:
int offset = needs_offset ? 4 : 0;
\end{lstlisting}

\begin{lstlisting}
// Ternary in Rust:
let offset = if needs_offset { 4 } else { 0 };
\end{lstlisting}

The Rust version of the ternary is not limited to a single Boolean comparison. Just as C and C++ can nest ternary operators, the \texttt{if}-block equivalent in Rust can have multiple \texttt{else if} branches as well:

\begin{lstlisting}
// Select an HTTP listening port
let http_port: u32 = if user_port > 1023 {
    user_port
} else if is_https {
    443
} else {
    80
};
\end{lstlisting}

In these sample listings is a particular syntactic feature of Rust: if the last expression of a block lacks a semicolon at the end, that value become the expression value of the block itself. So, in the above example all of \texttt{user\_port}, \texttt{443} and \texttt{80} are potential expression values of the \texttt{if}-block. The blocks they are in could have had many other statements prior to these lines, and the value of the expression would still be the line with no semicolon. Also, the declaration of \texttt{http\_port} includes a type specification so as to explicitly set the variable as being an unsigned 32-bit integer. Types are specified with a colon after the variable name and a type identifier following the colon.

While it is true that a C or C++ program could also use a multi-armed \texttt{if}-\texttt{else} construct in place of a nested ternary, the expression in Rust allows \texttt{http\_port} to be declared immutable from the outset. Were a C program to declare \texttt{http\_port} as \texttt{const}, it could not then assign it within the arms of the \texttt{if}-\texttt{else} construct.

Functions are defined with the \texttt{fn} keyword. Functions that return a value specify their return type between the closing parenthesis of the argument declaration and the opening curly bracket of the function body:

\begin{lstlisting}
// A basic factorial function:
fn factorial(n: u128) -> u128 {
    // Creates a range from 1 to n inclusive and
    // calculates the product of all the numbers.
    (1..=n).product()
}
\end{lstlisting}

Functions that do not return a value (the equivalent of \texttt{void} returns in C/C++) simply omit the return specification:

\begin{lstlisting}
// Toggle a Boolean value in-place:
fn toggle(value: &mut bool) {
    *value ^= true;
}
\end{lstlisting}

The remaining elements of syntax are close-enough to C as to not need specific mentioning.

\subsection{Traits and Generics}

Rust is a strongly typed and statically typed language, with traits and generic functions to achieve type polymorphism. Rust can also be considered an object-oriented language, as it meets the definition given in~\cite{design.patterns.1994} (quoted in~\cite[Chapter~17]{rpl.2019}):

\begin{quote}
An object packages both data and the procedures that operate on that data.
\end{quote}

While Rust does not explicitly have classes and inheritance, it does have the capability to attach functionality to data types (both structures and enumerations) through its traits system.

Defining a trait is a matter of giving it a name and listing the methods along with their type signatures:

\begin{lstlisting}
trait Quote {
    // Implementing Quote for a type means the values of
    // that type now have quoted() as a method.
    fn quoted(self) -> String;
}
\end{lstlisting}

Traits may be declared as extensions of other traits, in which case a type that implements the sub-trait must also implement the original trait:

\begin{lstlisting}
trait BlockQuote: Quote {
    // Implementing BlockQuote means values of that type
    // have a blockquoted() method, as well as the
    // quoted() method.
    fn blockquoted(self) -> String;
}
\end{lstlisting}

Traits are applied to types by way of the \texttt{impl} keyword:

\begin{lstlisting}
// Implement the Quote trait for the String type:
impl Quote for String {
    fn quoted(self) -> String {
        format!("\"{}\"", self)
    }
}
\end{lstlisting}

Generics are the other means of polymorphism in Rust. Generics can be applied to functions, to type declarations, and to traits as well. Using a syntax similar to C++ templates or Java generics, a generic function can be declared that will be applicable to multiple types:

\begin{lstlisting}
// A generics version of the quoted() function:
fn quoted<T: ToString>(input: T) -> String {
    format!("\"{}\"", input.to_string())
}
\end{lstlisting}

In this example, the function \texttt{quoted} can be called on any type of argument. However, the argument's type \textit{must} implement the \texttt{ToString} trait; this ensures that the call to the \texttt{to\_string()} method on line 3 will not fail. Note that a generic parameter may be bound to more than one trait, or to no traits at all.

Deciding whether to use trait objects or generic code is often based on subtle factors. The two features are both based on traits, so there is commonality between them. One notable difference is that Rust may have to compile a generic function several times, for each type it is used with. This can lead to larger executable code being produced by the compiler.

However, generics have three advantages over trait objects~\cite[Chapter~11]{programming.rust.2021}:

\begin{enumerate}
\item \textbf{Speed.} Types are specified at compile-time, either explicitly or through type inference. The enables the compiler to make optimization decisions when generating the code.
\item \textbf{Support.} Traits support features such as associated functions. Some of these features will only work with generics.
\item \textbf{Flexibility.} Generic type parameters can easily be bound to multiple traits, something trait objects cannot do. The sub-traits facility can be used to work around this, but it is more involved than the generics approach.
\end{enumerate}

\subsection{Ownership and Memory Management}

An area where Rust is distinct from other C-based languages is in the way it manages memory and tracks values on the stack and heap. Rust uses an ownership system~\cite[Chapter~4]{programming.rust.2021}, with the ability to specify lifetime information for reference types. There is no automated garbage collection, and resources are managed through a convention of \textit{resource acquisition is initialization}~\cite{cpp.design.evolution.1994} with optional reference counting. Rust's design for memory safety does not permit null pointers, dangling pointers, or data races.

With languages such as C and C++, \textit{data ownership} is handled largely through practice and convention. An instance of a C++ \texttt{std::string} owns the buffer allocated for the storage of the string data. Other variables may be created, though, that point to the same buffer or a single character within it. These other interests in the content of the string buffer have their own responsibility for noticing when the original string object is destroyed and the buffer freed. After such point, the outside interests are each responsible for marking their references as no longer valid.

In contrast, Rust integrates the concept of ownership directly into the language itself. Compile-time checks enforce ownership and report violations. When the owner of a value is ``dropped'' (Rust terminology for freeing) the owned value is dropped as well. While the variables themselves are on the stack, the content is allocated on the heap. Variables own their values, and the complex datatypes (structs, tuples, arrays, and vectors) own their elements.

For most types in Rust, an operation on a variable such as assignment or passing it to a function (or returning it from a function) does not copy the value: it moves the value. The current owner (``source'') gives up ownership of the value and the new owner (``destination'') assumes control of the value's lifetime. The source becomes uninitialized and is no longer usable. This is referred to as \textit{moving} the value. Rust does excuse the very simple types (integers, floating-point numbers and characters) from ownership rules; they are considered \texttt{Copy} types and have a trait called \texttt{Copy} associated with them.

Consider this simple example:

\begin{lstlisting}
let a: &String;
{
    let b = "Data".to_string();
    // This assigns a reference to b:
    a = &b;
}
// This line triggers a lifespan error:
println!("{}", a);
\end{lstlisting}

Here, the variable \texttt{a} is declared in a higher scope than the variable \texttt{b}. Thus, it survives the end-of-scope (line 6) that leads to \texttt{b} being dropped. Because line 8 attempts to make use of \texttt{a}, Rust's compiler steps in and declares that \texttt{b} does not live long-enough for this code to be valid. Similarly:

\begin{lstlisting}
let a: String;
let b = "Test".to_string();
a = b; // This moves the value of b. b is now invalid.
println!("{}", a);
println!("{}", b); // Error here
\end{lstlisting}

The assignment on line 3 moves the value of the string from \texttt{b} to \texttt{a}. While line 4 is perfectly fine, line 5 is the source of an error because \texttt{b} is no longer a valid value.

There is much more to the semantics of ownership and moving, particularly when pointers become involved. Both~\cite{rpl.2019} and~\cite{programming.rust.2021} go into greater depth on the topic.
